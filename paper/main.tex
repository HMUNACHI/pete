\documentclass{article}
\usepackage{style,times}
\usepackage{booktabs}
\usepackage{caption}
\usepackage{array}
\usepackage{graphicx}
\usepackage{array}
\usepackage{booktabs}
\usepackage{adjustbox}
\usepackage{amsmath, amsthm, amssymb}
\newtheorem{theorem}{Theorem}
\newtheorem{lemma}{Lemma}
\usepackage{hyperref}
\usepackage{url}
\begin{document}
\begin{center}
\title{Parameter-Efficient Transformer Embedding}
\end{center}
\author{Henry Ndubuaku \\ 
Cactus Compute, Inc. \\
London, United Kingdom \\
\texttt{henry@cactuscompute.com} \\
\And
Mouad Talhi \\
Department of Computing, Imperial College \\
London, United Kingdom \\
\texttt{mt924@ic.ac.uk} \\
}
\maketitle
\begin{abstract}
Embedding layers in transformer-based language models typically account for the largest share of model parameters, scaling with vocabulary size but not yielding performance gains linear to scale. We propose an alternative approach in which token embedding vectors are first generated deterministically, directly from the token IDs using a Fourier expansion of their normalized values, followed by a lightweight multilayer perceptron (MLP) that captures higher-order interactions. We train standard transformers with and our architecture on natural language inference tasks (SNLI and MNLI), and evaluate zero-shot performance on sentence textual similarity (STS-B). Our results demonstrate that the proposed method achieves competitive performance using significantly fewer parameters, trains faster, and generalizes well without the need for dropout. This proof-of-concept study highlights the potential for scalable, memory-efficient language models and motivates further large-scale experimentation on our posit. The code for reproducing and pre-trained weights are available at \url{https://github.com/HMUNACHI/pete}.
\end{abstract}
\section{Introduction and Related Work}
Embedding layers, which are typically represented as 2D matrices of dimensions \(V \times d\) (where \(V\) is the vocabulary size and \(d\) the embedding dimension), often contribute more parameters in linguistic transformer models than other layers. Despite this, they do not necessarily yield proportional performance gains~\cite{lan2019albert, rajbhandari2020zero, shen2020qbert}. Several factors are thought to contribute to this inefficiency. 

Firstly, sparsity issues can lead to under-optimized embeddings since rare tokens remain underrepresented during training~\cite{svenstrup2017hash}. More importantly, embeddings may induce redundancy by assigning dense vectors to tokens with overlapping semantic roles, thereby wasting capacity~\cite{lan2019albert}. Additionally, traditional embeddings do not exploit entropy-driven compression; they allocate excessive parameters to frequent tokens without addressing the inherent redundancy in token distributions~\cite{shu2017compressing}.

Numerous research efforts have aimed to alleviate this parameter inefficiency by employing compression, adaptive designs, and alternative representations. Cai et al. \cite{cai2023dimensionlifting} highlighted the inefficiency of high-dimensional embedding representations in knowledge graph embeddings, proposing the Dimension Lifting Network (LiftNet). LiftNet replaces wide embeddings with a narrow embedding layer followed by a dimension lifting network, though this approach is not directly generalizable to all use cases. Xu et al. \cite{xu2023tensorgpt} introduced TensorGPT, which leverages Tensor-Train Decomposition to reduce the embedding layer's parameter count by up to 38.4 times with minimal performance degradation. In TensorGPT, a pre-trained embedding is projected to a lower dimension to facilitate deployment, even though a large embedding table must still be initially trained. Yan et al. \cite{yan2021adaptivemaskedtwins} proposed an Adaptively-Masked Twins-based Layer that dynamically adjusts embedding dimensions based on feature values, achieving significant memory savings and improved parameter utilization, albeit with a more complex training regime. Wang et al. \cite{wang2020structured} presented Structured Embedding Compression, which uses matrix factorization and product quantization to reduce the parameter count. Similarly, Lan et al. \cite{lan2019albert} developed ALBERT, a lightweight model that ties embeddings with the transformer's hidden representations to reduce redundancy. Shen et al. \cite{shen2020qbert} introduced Q-BERT, which employs Hessian-based quantization for embedding layers, though this method is compute-intensive. Additionally, hash-based embeddings have shown promise; for instance, Svenstrup et al. \cite{svenstrup2017hash} replaced traditional embedding tables with hash embeddings, achieving substantial parameter reductions for large vocabularies, albeit with specialized training procedures. Furthermore, although various forms of weight tying (such as embedding-output weight tying) are used in state-of-the-art models for language generation, such techniques do not readily benefit classification and language understanding tasks where output layers typically do not match the dimensions of the embedding layers. Also, sub-word tokenizations limits vocabulary size explosion, but we rather propose a solution where the vocabulary size is not an algorithmic complexity.

Despite these advances, further exploration is required to develop universally efficient and effective embedding mechanisms. In this work, we propose approximating the embedding layer with a combination of analytical methods and fewer learned parameters, opposed to learning embedding vectors for each token in the vocabulary. First motivation; Embedding layers in transformer models are fully factorized projections with not token inter-dependence, the semantic and pragmatic relationships are learned in the attention blocks. Second motivation; Byte-Pair Encoding (BPE) assigns token IDs based on each token's frequency in the corpus, effectively following a statistical pattern. The token ids carry distinguishing information. To this end, we argue that legacy embedding layers could be approximated by first transforming discrete token IDs into continuous values in the range \([-1, 1]\) (described in the next section), computing their Fourier expansions up to a predetermined degree \(n\) which corresponds to the desired embedding dimension, then projecting unto a more aligned vector space with a shared Multi-Layered Perceptron. Other polynomial bases (e.g., Chebyshev, Legendre, Taylor) could be employed, however Taylor polynomials require the computation of derivatives, Chebyshev polynomials exhibit auto-regressive properties, and Legendre polynomials involve factorial computations. These make them less amenable to optimizations on accelerators such as GPUs, TPUs and NPUs. Computing Fourier on the fly at inference is still computationally expensive compared to merely mapping token IDs to learned vectors, however they are very friendly hand-crafted hard-ware aware implementations, which we did in this work. 
\section{Methodology}
We adopt a parameter-efficient strategy to encode tokens by leveraging fixed Fourier basis functions combined with a learnable multilayer perceptron (MLP). This design is motivated by an information-theoretic perspective. In particular, BPE tokenization splits the input text into statistically significant word/subword units that are \emph{entropy-efficient} in terms of compression~\cite{sennrich2016neural}. Recall that for a token \(t\) with probability \(p(t)\), the \emph{surprisal} (self-information) is given by
\[
I(t) = -\log p(t),
\]
so that rare tokens (with low \(p(t)\)) contribute higher information. In many BPE schemes, token IDs are assigned in frequency order, meaning that frequent tokens receive lower IDs while rare tokens are assigned higher IDs. Although the token ID itself is an arbitrary label, its ordering reflects statistical properties of the vocabulary.
To harness this structure, we first normalize a token's integer ID \(p\) into the continuous interval \([-1,1]\) by defining
\[
x \;=\; 2\,\frac{p}{\texttt{vocabulary\_size}-1} \;-\; 1.
\]
This normalization maps discrete token IDs into a continuous, scale-invariant domain, allowing subsequent smooth transformations. Importantly, while the mapping is deterministic, it preserves the relative differences among tokens so that minor alterations (e.g., a word swap) affect the vector's magnitude more than its overall direction.
Next, we expand \(x\) into a high-dimensional embedding using a Fourier basis. For a chosen model dimension \(d_{\mathrm{model}}\), the token embedding \(\mathbf{T}(p) \in \mathbb{R}^{d_{\mathrm{model}}}\) is defined component-wise as
\[
T_{i}(p) 
\;=\;
\begin{cases}
\sin\Bigl(\bigl(\lfloor i/2 \rfloor + 1\bigr)\,\pi\,x\Bigr), & \text{if } i \text{ is even},\\[6pt]
\cos\Bigl(\bigl(\lfloor i/2 \rfloor + 1\bigr)\,\pi\,x\Bigr), & \text{if } i \text{ is odd}.
\end{cases}
\]
Here, lower-order Fourier terms (e.g., \(T_0(x)\) and \(T_1(x)\)) capture global, coarse-grained information, while higher-order terms (\(T_n(x)\) for \(n\geq2\)) encode finer details. Under the small-angle approximation, the difference between embeddings of adjacent tokens (i.e., \(p\) and \(p+1\)) is approximately
\[
\Delta x \;=\; \frac{2}{V-1},
\]
so that with a large vocabulary \(V\), adjacent token embeddings in raw Fourier space lie very close. Moreover, because BPE token assignment is not strictly semantically monotonic (e.g., the tokens ``cat'' and ``cathedral'' might receive consecutive IDs despite semantic differences), the Fourier expansion alone may not sufficiently differentiate tokens with similar IDs.
To mitigate this, we append a learnable MLP to the Fourier features. The final token representation is given by
\[
\mathbf{E}(p) \;=\; \text{MLP}\bigl(\mathbf{T}(p)\bigr) + \mathbf{T}(p).
\]
The role of the MLP is to \emph{stretch apart} tokens whose initial Fourier representations are too similar, ensuring that semantically distinct tokens (even if numerically adjacent) are mapped to adequately separated vectors for the attention mechanism. From the universal approximation viewpoint, given any continuous target embedding function \(f: [-1,1] \to \mathbb{R}\), the MLP can approximate the residual function \(H(z)=f\bigl(\varphi^{-1}(z)\bigr)-z\) (where \(\varphi(x)=\mathbf{T}(p)\)) uniformly. That is, for every \(\epsilon>0\) there exists an MLP \(M\) such that
\[
\sup_{x \in [-1, 1]} \Bigl|\, M\bigl(\varphi(x)\bigr) + \varphi(x) - f(x) \Bigr| < \epsilon.
\]
An information-theoretic perspective further illuminates this design via the Partial Information Decomposition (PID) framework~\cite{williams2010nonnegative, bertschinger2013quantifying}. In this view, the Fourier components of \(\mathbf{T}(p)\) serve as \emph{sources} of information about the token. Specifically:
\begin{itemize}
    \item \textbf{Unique Information} is carried directly by the Fourier expansion \(\mathbf{T}(p)\).
    \item \textbf{Redundant Information} represents overlapping aspects that appear both in \(\mathbf{T}(p)\) and in the MLP-transformed output.
    \item \textbf{Synergistic Information} emerges only when combining the Fourier features via the non-linear MLP.
\end{itemize}
The MLP is crucial for extracting higher-order interactions that a mere linear mapping could not achieve.
Finally, although dropout is commonly used to reduce overfitting, we observed that applying dropout to these normalized continuous mappings disrupts the smooth progression of token IDs and degrades performance as expected in Polynomial-based approximations. The continuous normalization itself appears to provide a regularizing effect, contributing to the minimal overfitting observed even in over-parameterized regimes.
\section{Experimentation and Results}
Due to resource constraints in both compute (a single Nvidia RTX 4090), team size and time limitations, our experiments are intentionally scaled down to a proof-of-concept design, and not optimized for main tracks at top conferences. Different neural architectures require different optimal hyper-parameters, but we evaluate the proposed transformer with Fourier embeddings (denoted as Fourier embeddings henceforth) and the corresponding baseline transformer under identical settings (established setups for transformers), differing only in the embedding layer. In the baseline transformer, the token embeddings are learned conventionally. Although extensive pre-training and evaluation at much larger model sized, on more benchmarks (e.g., GLUE or specialized reasoning tasks) would provide deeper insights, we defer these directions to future work. Our focus here is to assess how effectively an attention-based model can learn semantic information using a semi-approximated embedding mechanism.

Our experimental setup employs mixed-precision training on the entailment subsets of the SNLI and MNLI datasets~\cite{snli,mnli} with a batch size of 128 over 122,700 iterations, a learning rate of \(2\times10^{-5}\), and 1,000 warmup steps. For the baseline transformer, we use a dropout probability of 0.1 (yeiled best results), whereas Fourier embedding omits dropout. Fourier embedding implements its embedding mechanism via a custom CUDA kernel that fuses normalization and Fourier expansion. We utilize the BERT Tokenizer~\cite{devlin2019bert} (vocabulary size 30,522), rotary positional encoding~\cite{su2021rotary}, root-mean-squared layer normalization~\cite{liu2020rmsnorm}, and GeGLU activation~\cite{shazeer2020glu}. In addition, we employ average pooling and the AdamW optimizer~\cite{loshchilov2019decoupled}. Following the Fourier expansion, we include a position-wise feed-forward block with an intermediate expansion factor of 4 and GeGLU activation. While a large MLP might nearly match the parameter count of a learned embedding matrix, our experiments indicate that replacing this block with a simple linear layer only marginally degrades performance, while further reducing model size. Our primary objective is to isolate the impact of substituting the learned embedding layer with a deterministic alternative.

For training, we employ a contrastive loss function inspired by CLIP~\cite{radford2021learning} and InfoNCE~\cite{oord2018representation}. This loss encourages embeddings corresponding to matching sentence pairs (or entailment pairs) to be close in the embedding space, while non-matching pairs are pushed apart. A learnable temperature parameter is used to appropriately scale the cosine similarity scores. This contrastive framework leverages the entailment data (approximately 200k samples) to enforce semantic consistency in the learned representations.
Table~\ref{tab:model_scores} below summarizes our main findings.
\begin{table}[h!]
    \centering
    \begin{adjustbox}{max width=\textwidth}
    \begin{tabular}{|c|c|c|c|c|c|c|}
        \hline
        \textbf{Model} & \textbf{Layers/Heads} & \textbf{d-model} & \textbf{Params} & \textbf{STSB Spearman-R} & \textbf{STSB Pearson-R} & \textbf{Training Time}\\
        \hline
        Transformer (Fourier Embedding) & 1 & 256 & 1.1m & 74.93 & 74.54 & 37.88 min\\
        Transformer & 1 & 256 & 8.9m & 77.01 & 76.80 & 48.48 min\\
        \hline
        Transformer (Fourier Embedding) & 1 & 512 & 4.7m & 75.21 & 74.65 & 1.349 hr\\
        Transformer & 1 & 512 & 20.1m & 77.50 & 76.78 & 1.688 hr\\
        \hline
        Transformer (Fourier Embedding) & 2 & 256 & 2.2m & 76.38 & 76.02 & 1.009 hr\\
        Transformer & 2 & 256 & 9.9m & 77.34 & 76.89 & 1.322 hr\\
        \hline
        Transformer (Fourier Embedding) & 2 & 512 & 8.9m & 77.40 & 77.11 & 2.27 hr\\
        Transformer & 2 & 512 & 24.3m & 77.54 & 76.96 & 2.982 hr\\
        \hline
    \end{tabular}
    \end{adjustbox}
    \caption{STS-B validation scores without fine-tuning on STS-B.}
    \label{tab:model_scores}
\end{table}

With sufficient capacity (i.e., an adequately chosen \(d_{\text{model}}\) and number of layers), the Fourier-based embedding can yield representations as effective as those of learned embeddings. This finding supports the claim that semantic information can be efficiently captured by a deterministic, parameter-free mapping when combined with appropriate downstream processing. In particular, our results show that at two layers/heads with dimensions of 256 or 512, Fourier embedding converges to performance levels comparable to a standard transformer. By contrast, transformers of the same size appear over-parameterized, offering no measurable performance gains while introducing unnecessary parameters. Conversely, models quickly become under-parameterized when they use fewer than two layers/heads or a \(d_{\text{model}}\) below 256. Another advantage of Fourier embedding is its avoidance of the computational overhead associated with large embedding tables. Our custom CUDA kernel, which fuses normalization and Fourier expansion, likely contributes to reduced training times. 

While these improvements are not strictly proportional to model size, they may become substantial in large language-modeling scenarios. Moreover, Fourier embedding allows for flexible downscaling without sacrificing the capacity to learn meaningful relationships, since it does not rely on the vocabulary size. When parameter counts are held constant, Fourier embedding offers notably better performance than transformers by freeing up capacity for deeper attention layers. Traditional transformers allocate a large fraction of their parameters to the embedding layer, which grows with the vocabulary size. In contrast, our method replaces this with a deterministic Fourier expansion, combined with a relatively small MLP, eliminating the need for a heavy, learned embedding table. This shift effectively "frees up" parameters that can then be invested elsewhere in the network, leading to more balanced and efficient architectures.

Ultimately, parameter size percentage reductions slows down as transformer layers are scaled horizontally (adding more layers), hence we experimented with further parameter reduction by transforming the intermediate dimensions of the MLP blocks from dim x 4 to dim / 4, equivalent to using lower rank weight matrices. Table~\ref{tab:model_comparisons} below summarizes the performance of these much smaller Parameter-Efficient Transformer Embeddings (PETE) when trained with the same regime and fine-tuned on STS-B.

\begin{table}[h!]
    \centering
    \begin{adjustbox}{max width=\textwidth}
    \begin{tabular}{|c|c|c|c|c|c|c|}
        \hline
        \textbf{Model} & \textbf{Params} & \textbf{STSB Pearson-R} & \textbf{STSB Spearman-R}\\
        \hline
        PETE & 58k+ & 69.0 & 69.5\\
        \hline
        PETE & 396k+ & 78.0 & 78.0\\
        \hline
        PETE & 1.5m+ & 79.7 & 79.7\\
        \hline
        PETE & 3.6m+ & \textbf{81.7} & \textbf{81.9}\\
        \hline
        BERT-Tiny (official report) & 4m+ & 74.3 & 73.6\\
        \hline
        BERT-Mini (official report) & 11m+ & 74.3 & 73.6\\
        \hline
        TinyBERT (official report) & 14.5m & -- & 80.4\\
        \hline 
        MobileBert-Tiny (official report) & 15.1m  &  -- & 80.1\\
        \hline
        BERT-Small (official report) & 29m+ & 78.8 & 77.0\\
        \hline
    \end{tabular}
    \end{adjustbox}
    \caption{STS-B validation scores after fine-tuning on STS-B against tiny models}
    \label{tab:model_comparisons}
\end{table}
\section{Discussion and Conclusion}
The experimental results support our central claim: a Fourier-based, parameter-free embedding scheme can yield competitive performance compared to traditional learned embeddings while substantially reducing both the parameter count and training time. In configurations with equal overall parameters, the PETE even outperforms the traditional transformer, owing to its efficient allocation of resources. By eliminating large, learned embedding tables, our method frees up capacity for deeper attention layers and better overall parameter utilization.

It is important to note that these experiments were conducted on relatively small-scale models and focused on natural language inference and the STS-B task. Whether the benefits extend to larger-scale models and more diverse tasks such as large-scale reasoning remains an open question for future work. In particular, scaling up to vocabularies in the hundreds of thousands may pose challenges, as a dense coverage of the continuous interval could lead to near-colliding embeddings. Future research should investigate whether attention blocks can effectively separate these near-colliding embeddings and whether the deterministic Fourier expansion can capture discrete lexical phenomena such as morphological variants, homonyms, or polysemy. Another promising direction is the exploration of alternative polynomial bases (e.g., Chebyshev or Legendre) for embedding mechanisms. Although these alternatives may offer theoretical advantages, our findings suggest that the Fourier basis is particularly well-suited for optimization on modern accelerators such as GPUs and TPUs. Its analytic form not only contributes to memory and computational efficiency but also opens avenues for improved interpretability of token embeddings. For example, by examining how final representations evolve as a function of token ID, one may gain insights into the model's internal semantic organization.
In conclusion, our work demonstrates that a deterministic, Fourier-based token embedding, when paired with appropriate downstream processing, can serve as an efficient alternative to learned embedding tables. This approach not only reduces the parameter burden but also reallocates capacity more effectively within the network, potentially leading to more robust and scalable models.
\documentclass{article}
\usepackage{style,times}
\usepackage{booktabs}
\usepackage{caption}
\usepackage{array}
\usepackage{graphicx}
\usepackage{array}
\usepackage{booktabs}
\usepackage{adjustbox}
\usepackage{amsmath, amsthm, amssymb}
\newtheorem{theorem}{Theorem}
\newtheorem{lemma}{Lemma}
\usepackage{hyperref}
\usepackage{url}
\begin{document}
\begin{center}
\title{Parameter-Efficient Transformer Embedding}
\end{center}
\author{Henry Ndubuaku \\ 
Cactus Compute, Inc. \\
London, United Kingdom \\
\texttt{henry@cactuscompute.com} \\
\And
Mouad Talhi \\
Department of Computing, Imperial College \\
London, United Kingdom \\
\texttt{mt924@ic.ac.uk} \\
}
\maketitle
\begin{abstract}
Embedding layers in transformer-based language models typically account for the largest share of model parameters, scaling with vocabulary size but not yielding performance gains linear to scale. We propose an alternative approach in which token embedding vectors are first generated deterministically, directly from the token IDs using a Fourier expansion of their normalized values, followed by a lightweight multilayer perceptron (MLP) that captures higher-order interactions. We train standard transformers with and our architecture on natural language inference tasks (SNLI and MNLI), and evaluate zero-shot performance on sentence textual similarity (STS-B). Our results demonstrate that the proposed method achieves competitive performance using significantly fewer parameters, trains faster, and generalizes well without the need for dropout. This proof-of-concept study highlights the potential for scalable, memory-efficient language models and motivates further large-scale experimentation on our posit. The code for reproducing and pre-trained weights are available at \url{https://github.com/HMUNACHI/pete}.
\end{abstract}
\section{Introduction and Related Work}
Embedding layers, which are typically represented as 2D matrices of dimensions \(V \times d\) (where \(V\) is the vocabulary size and \(d\) the embedding dimension), often contribute more parameters in linguistic transformer models than other layers. Despite this, they do not necessarily yield proportional performance gains~\cite{lan2019albert, rajbhandari2020zero, shen2020qbert}. Several factors are thought to contribute to this inefficiency. 

Firstly, sparsity issues can lead to under-optimized embeddings since rare tokens remain underrepresented during training~\cite{svenstrup2017hash}. More importantly, embeddings may induce redundancy by assigning dense vectors to tokens with overlapping semantic roles, thereby wasting capacity~\cite{lan2019albert}. Additionally, traditional embeddings do not exploit entropy-driven compression; they allocate excessive parameters to frequent tokens without addressing the inherent redundancy in token distributions~\cite{shu2017compressing}.

Numerous research efforts have aimed to alleviate this parameter inefficiency by employing compression, adaptive designs, and alternative representations. Cai et al. \cite{cai2023dimensionlifting} highlighted the inefficiency of high-dimensional embedding representations in knowledge graph embeddings, proposing the Dimension Lifting Network (LiftNet). LiftNet replaces wide embeddings with a narrow embedding layer followed by a dimension lifting network, though this approach is not directly generalizable to all use cases. Xu et al. \cite{xu2023tensorgpt} introduced TensorGPT, which leverages Tensor-Train Decomposition to reduce the embedding layer's parameter count by up to 38.4 times with minimal performance degradation. In TensorGPT, a pre-trained embedding is projected to a lower dimension to facilitate deployment, even though a large embedding table must still be initially trained. Yan et al. \cite{yan2021adaptivemaskedtwins} proposed an Adaptively-Masked Twins-based Layer that dynamically adjusts embedding dimensions based on feature values, achieving significant memory savings and improved parameter utilization, albeit with a more complex training regime. Wang et al. \cite{wang2020structured} presented Structured Embedding Compression, which uses matrix factorization and product quantization to reduce the parameter count. Similarly, Lan et al. \cite{lan2019albert} developed ALBERT, a lightweight model that ties embeddings with the transformer's hidden representations to reduce redundancy. Shen et al. \cite{shen2020qbert} introduced Q-BERT, which employs Hessian-based quantization for embedding layers, though this method is compute-intensive. Additionally, hash-based embeddings have shown promise; for instance, Svenstrup et al. \cite{svenstrup2017hash} replaced traditional embedding tables with hash embeddings, achieving substantial parameter reductions for large vocabularies, albeit with specialized training procedures. Furthermore, although various forms of weight tying (such as embedding-output weight tying) are used in state-of-the-art models for language generation, such techniques do not readily benefit classification and language understanding tasks where output layers typically do not match the dimensions of the embedding layers. Also, sub-word tokenizations limits vocabulary size explosion, but we rather propose a solution where the vocabulary size is not an algorithmic complexity.

Despite these advances, further exploration is required to develop universally efficient and effective embedding mechanisms. In this work, we propose approximating the embedding layer with a combination of analytical methods and fewer learned parameters, opposed to learning embedding vectors for each token in the vocabulary. First motivation; Embedding layers in transformer models are fully factorized projections with not token inter-dependence, the semantic and pragmatic relationships are learned in the attention blocks. Second motivation; Byte-Pair Encoding (BPE) assigns token IDs based on each token's frequency in the corpus, effectively following a statistical pattern. The token ids carry distinguishing information. To this end, we argue that legacy embedding layers could be approximated by first transforming discrete token IDs into continuous values in the range \([-1, 1]\) (described in the next section), computing their Fourier expansions up to a predetermined degree \(n\) which corresponds to the desired embedding dimension, then projecting unto a more aligned vector space with a shared Multi-Layered Perceptron. Other polynomial bases (e.g., Chebyshev, Legendre, Taylor) could be employed, however Taylor polynomials require the computation of derivatives, Chebyshev polynomials exhibit auto-regressive properties, and Legendre polynomials involve factorial computations. These make them less amenable to optimizations on accelerators such as GPUs, TPUs and NPUs. Computing Fourier on the fly at inference is still computationally expensive compared to merely mapping token IDs to learned vectors, however they are very friendly hand-crafted hard-ware aware implementations, which we did in this work. 
\section{Methodology}
We adopt a parameter-efficient strategy to encode tokens by leveraging fixed Fourier basis functions combined with a learnable multilayer perceptron (MLP). This design is motivated by an information-theoretic perspective. In particular, BPE tokenization splits the input text into statistically significant word/subword units that are \emph{entropy-efficient} in terms of compression~\cite{sennrich2016neural}. Recall that for a token \(t\) with probability \(p(t)\), the \emph{surprisal} (self-information) is given by
\[
I(t) = -\log p(t),
\]
so that rare tokens (with low \(p(t)\)) contribute higher information. In many BPE schemes, token IDs are assigned in frequency order, meaning that frequent tokens receive lower IDs while rare tokens are assigned higher IDs. Although the token ID itself is an arbitrary label, its ordering reflects statistical properties of the vocabulary.
To harness this structure, we first normalize a token's integer ID \(p\) into the continuous interval \([-1,1]\) by defining
\[
x \;=\; 2\,\frac{p}{\texttt{vocabulary\_size}-1} \;-\; 1.
\]
This normalization maps discrete token IDs into a continuous, scale-invariant domain, allowing subsequent smooth transformations. Importantly, while the mapping is deterministic, it preserves the relative differences among tokens so that minor alterations (e.g., a word swap) affect the vector's magnitude more than its overall direction.
Next, we expand \(x\) into a high-dimensional embedding using a Fourier basis. For a chosen model dimension \(d_{\mathrm{model}}\), the token embedding \(\mathbf{T}(p) \in \mathbb{R}^{d_{\mathrm{model}}}\) is defined component-wise as
\[
T_{i}(p) 
\;=\;
\begin{cases}
\sin\Bigl(\bigl(\lfloor i/2 \rfloor + 1\bigr)\,\pi\,x\Bigr), & \text{if } i \text{ is even},\\[6pt]
\cos\Bigl(\bigl(\lfloor i/2 \rfloor + 1\bigr)\,\pi\,x\Bigr), & \text{if } i \text{ is odd}.
\end{cases}
\]
Here, lower-order Fourier terms (e.g., \(T_0(x)\) and \(T_1(x)\)) capture global, coarse-grained information, while higher-order terms (\(T_n(x)\) for \(n\geq2\)) encode finer details. Under the small-angle approximation, the difference between embeddings of adjacent tokens (i.e., \(p\) and \(p+1\)) is approximately
\[
\Delta x \;=\; \frac{2}{V-1},
\]
so that with a large vocabulary \(V\), adjacent token embeddings in raw Fourier space lie very close. Moreover, because BPE token assignment is not strictly semantically monotonic (e.g., the tokens ``cat'' and ``cathedral'' might receive consecutive IDs despite semantic differences), the Fourier expansion alone may not sufficiently differentiate tokens with similar IDs.
To mitigate this, we append a learnable MLP to the Fourier features. The final token representation is given by
\[
\mathbf{E}(p) \;=\; \text{MLP}\bigl(\mathbf{T}(p)\bigr) + \mathbf{T}(p).
\]
The role of the MLP is to \emph{stretch apart} tokens whose initial Fourier representations are too similar, ensuring that semantically distinct tokens (even if numerically adjacent) are mapped to adequately separated vectors for the attention mechanism. From the universal approximation viewpoint, given any continuous target embedding function \(f: [-1,1] \to \mathbb{R}\), the MLP can approximate the residual function \(H(z)=f\bigl(\varphi^{-1}(z)\bigr)-z\) (where \(\varphi(x)=\mathbf{T}(p)\)) uniformly. That is, for every \(\epsilon>0\) there exists an MLP \(M\) such that
\[
\sup_{x \in [-1, 1]} \Bigl|\, M\bigl(\varphi(x)\bigr) + \varphi(x) - f(x) \Bigr| < \epsilon.
\]
An information-theoretic perspective further illuminates this design via the Partial Information Decomposition (PID) framework~\cite{williams2010nonnegative, bertschinger2013quantifying}. In this view, the Fourier components of \(\mathbf{T}(p)\) serve as \emph{sources} of information about the token. Specifically:
\begin{itemize}
    \item \textbf{Unique Information} is carried directly by the Fourier expansion \(\mathbf{T}(p)\).
    \item \textbf{Redundant Information} represents overlapping aspects that appear both in \(\mathbf{T}(p)\) and in the MLP-transformed output.
    \item \textbf{Synergistic Information} emerges only when combining the Fourier features via the non-linear MLP.
\end{itemize}
The MLP is crucial for extracting higher-order interactions that a mere linear mapping could not achieve.
Finally, although dropout is commonly used to reduce overfitting, we observed that applying dropout to these normalized continuous mappings disrupts the smooth progression of token IDs and degrades performance as expected in Polynomial-based approximations. The continuous normalization itself appears to provide a regularizing effect, contributing to the minimal overfitting observed even in over-parameterized regimes.
\section{Experimentation and Results}
Due to resource constraints in both compute (a single Nvidia RTX 4090), team size and time limitations, our experiments are intentionally scaled down to a proof-of-concept design, and not optimized for main tracks at top conferences. Different neural architectures require different optimal hyper-parameters, but we evaluate the proposed transformer with Fourier embeddings (denoted as Fourier embeddings henceforth) and the corresponding baseline transformer under identical settings (established setups for transformers), differing only in the embedding layer. In the baseline transformer, the token embeddings are learned conventionally. Although extensive pre-training and evaluation at much larger model sized, on more benchmarks (e.g., GLUE or specialized reasoning tasks) would provide deeper insights, we defer these directions to future work. Our focus here is to assess how effectively an attention-based model can learn semantic information using a semi-approximated embedding mechanism.

Our experimental setup employs mixed-precision training on the entailment subsets of the SNLI and MNLI datasets~\cite{snli,mnli} with a batch size of 128 over 122,700 iterations, a learning rate of \(2\times10^{-5}\), and 1,000 warmup steps. For the baseline transformer, we use a dropout probability of 0.1 (yeiled best results), whereas Fourier embedding omits dropout. Fourier embedding implements its embedding mechanism via a custom CUDA kernel that fuses normalization and Fourier expansion. We utilize the BERT Tokenizer~\cite{devlin2019bert} (vocabulary size 30,522), rotary positional encoding~\cite{su2021rotary}, root-mean-squared layer normalization~\cite{liu2020rmsnorm}, and GeGLU activation~\cite{shazeer2020glu}. In addition, we employ average pooling and the AdamW optimizer~\cite{loshchilov2019decoupled}. Following the Fourier expansion, we include a position-wise feed-forward block with an intermediate expansion factor of 4 and GeGLU activation. While a large MLP might nearly match the parameter count of a learned embedding matrix, our experiments indicate that replacing this block with a simple linear layer only marginally degrades performance, while further reducing model size. Our primary objective is to isolate the impact of substituting the learned embedding layer with a deterministic alternative.

For training, we employ a contrastive loss function inspired by CLIP~\cite{radford2021learning} and InfoNCE~\cite{oord2018representation}. This loss encourages embeddings corresponding to matching sentence pairs (or entailment pairs) to be close in the embedding space, while non-matching pairs are pushed apart. A learnable temperature parameter is used to appropriately scale the cosine similarity scores. This contrastive framework leverages the entailment data (approximately 200k samples) to enforce semantic consistency in the learned representations.
Table~\ref{tab:model_scores} below summarizes our main findings.
\begin{table}[h!]
    \centering
    \begin{adjustbox}{max width=\textwidth}
    \begin{tabular}{|c|c|c|c|c|c|c|}
        \hline
        \textbf{Model} & \textbf{Layers/Heads} & \textbf{d-model} & \textbf{Params} & \textbf{STSB Spearman-R} & \textbf{STSB Pearson-R} & \textbf{Training Time}\\
        \hline
        Transformer (Fourier Embedding) & 1 & 256 & 1.1m & 74.93 & 74.54 & 37.88 min\\
        Transformer & 1 & 256 & 8.9m & 77.01 & 76.80 & 48.48 min\\
        \hline
        Transformer (Fourier Embedding) & 1 & 512 & 4.7m & 75.21 & 74.65 & 1.349 hr\\
        Transformer & 1 & 512 & 20.1m & 77.50 & 76.78 & 1.688 hr\\
        \hline
        Transformer (Fourier Embedding) & 2 & 256 & 2.2m & 76.38 & 76.02 & 1.009 hr\\
        Transformer & 2 & 256 & 9.9m & 77.34 & 76.89 & 1.322 hr\\
        \hline
        Transformer (Fourier Embedding) & 2 & 512 & 8.9m & 77.40 & 77.11 & 2.27 hr\\
        Transformer & 2 & 512 & 24.3m & 77.54 & 76.96 & 2.982 hr\\
        \hline
    \end{tabular}
    \end{adjustbox}
    \caption{STS-B validation scores without fine-tuning on STS-B.}
    \label{tab:model_scores}
\end{table}

With sufficient capacity (i.e., an adequately chosen \(d_{\text{model}}\) and number of layers), the Fourier-based embedding can yield representations as effective as those of learned embeddings. This finding supports the claim that semantic information can be efficiently captured by a deterministic, parameter-free mapping when combined with appropriate downstream processing. In particular, our results show that at two layers/heads with dimensions of 256 or 512, Fourier embedding converges to performance levels comparable to a standard transformer. By contrast, transformers of the same size appear over-parameterized, offering no measurable performance gains while introducing unnecessary parameters. Conversely, models quickly become under-parameterized when they use fewer than two layers/heads or a \(d_{\text{model}}\) below 256. Another advantage of Fourier embedding is its avoidance of the computational overhead associated with large embedding tables. Our custom CUDA kernel, which fuses normalization and Fourier expansion, likely contributes to reduced training times. 

While these improvements are not strictly proportional to model size, they may become substantial in large language-modeling scenarios. Moreover, Fourier embedding allows for flexible downscaling without sacrificing the capacity to learn meaningful relationships, since it does not rely on the vocabulary size. When parameter counts are held constant, Fourier embedding offers notably better performance than transformers by freeing up capacity for deeper attention layers. Traditional transformers allocate a large fraction of their parameters to the embedding layer, which grows with the vocabulary size. In contrast, our method replaces this with a deterministic Fourier expansion, combined with a relatively small MLP, eliminating the need for a heavy, learned embedding table. This shift effectively "frees up" parameters that can then be invested elsewhere in the network, leading to more balanced and efficient architectures.

Ultimately, parameter size percentage reductions slows down as transformer layers are scaled horizontally (adding more layers), hence we experimented with further parameter reduction by transforming the intermediate dimensions of the MLP blocks from dim x 4 to dim / 4, equivalent to using lower rank weight matrices. Table~\ref{tab:model_comparisons} below summarizes the performance of these much smaller Parameter-Efficient Transformer Embeddings (PETE) when trained with the same regime and fine-tuned on STS-B.

\begin{table}[h!]
    \centering
    \begin{adjustbox}{max width=\textwidth}
    \begin{tabular}{|c|c|c|c|c|c|c|}
        \hline
        \textbf{Model} & \textbf{Params} & \textbf{STSB Pearson-R} & \textbf{STSB Spearman-R}\\
        \hline
        PETE & 58k+ & 69.0 & 69.5\\
        \hline
        PETE & 396k+ & 78.0 & 78.0\\
        \hline
        PETE & 1.5m+ & 79.7 & 79.7\\
        \hline
        PETE & 3.6m+ & \textbf{81.7} & \textbf{81.9}\\
        \hline
        BERT-Tiny (official report) & 4m+ & 74.3 & 73.6\\
        \hline
        BERT-Mini (official report) & 11m+ & 74.3 & 73.6\\
        \hline
        TinyBERT (official report) & 14.5m & -- & 80.4\\
        \hline 
        MobileBert-Tiny (official report) & 15.1m  &  -- & 80.1\\
        \hline
        BERT-Small (official report) & 29m+ & 78.8 & 77.0\\
        \hline
    \end{tabular}
    \end{adjustbox}
    \caption{STS-B validation scores after fine-tuning on STS-B against tiny models}
    \label{tab:model_comparisons}
\end{table}
\section{Discussion and Conclusion}
The experimental results support our central claim: a Fourier-based, parameter-free embedding scheme can yield competitive performance compared to traditional learned embeddings while substantially reducing both the parameter count and training time. In configurations with equal overall parameters, the PETE even outperforms the traditional transformer, owing to its efficient allocation of resources. By eliminating large, learned embedding tables, our method frees up capacity for deeper attention layers and better overall parameter utilization.

It is important to note that these experiments were conducted on relatively small-scale models and focused on natural language inference and the STS-B task. Whether the benefits extend to larger-scale models and more diverse tasks such as large-scale reasoning remains an open question for future work. In particular, scaling up to vocabularies in the hundreds of thousands may pose challenges, as a dense coverage of the continuous interval could lead to near-colliding embeddings. Future research should investigate whether attention blocks can effectively separate these near-colliding embeddings and whether the deterministic Fourier expansion can capture discrete lexical phenomena such as morphological variants, homonyms, or polysemy. Another promising direction is the exploration of alternative polynomial bases (e.g., Chebyshev or Legendre) for embedding mechanisms. Although these alternatives may offer theoretical advantages, our findings suggest that the Fourier basis is particularly well-suited for optimization on modern accelerators such as GPUs and TPUs. Its analytic form not only contributes to memory and computational efficiency but also opens avenues for improved interpretability of token embeddings. For example, by examining how final representations evolve as a function of token ID, one may gain insights into the model's internal semantic organization.
In conclusion, our work demonstrates that a deterministic, Fourier-based token embedding, when paired with appropriate downstream processing, can serve as an efficient alternative to learned embedding tables. This approach not only reduces the parameter burden but also reallocates capacity more effectively within the network, potentially leading to more robust and scalable models.
\documentclass{article}
\usepackage{style,times}
\usepackage{booktabs}
\usepackage{caption}
\usepackage{array}
\usepackage{graphicx}
\usepackage{array}
\usepackage{booktabs}
\usepackage{adjustbox}
\usepackage{amsmath, amsthm, amssymb}
\newtheorem{theorem}{Theorem}
\newtheorem{lemma}{Lemma}
\usepackage{hyperref}
\usepackage{url}
\begin{document}
\begin{center}
\title{Parameter-Efficient Transformer Embedding}
\end{center}
\author{Henry Ndubuaku \\ 
Cactus Compute, Inc. \\
London, United Kingdom \\
\texttt{henry@cactuscompute.com} \\
\And
Mouad Talhi \\
Department of Computing, Imperial College \\
London, United Kingdom \\
\texttt{mt924@ic.ac.uk} \\
}
\maketitle
\begin{abstract}
Embedding layers in transformer-based language models typically account for the largest share of model parameters, scaling with vocabulary size but not yielding performance gains linear to scale. We propose an alternative approach in which token embedding vectors are first generated deterministically, directly from the token IDs using a Fourier expansion of their normalized values, followed by a lightweight multilayer perceptron (MLP) that captures higher-order interactions. We train standard transformers with and our architecture on natural language inference tasks (SNLI and MNLI), and evaluate zero-shot performance on sentence textual similarity (STS-B). Our results demonstrate that the proposed method achieves competitive performance using significantly fewer parameters, trains faster, and generalizes well without the need for dropout. This proof-of-concept study highlights the potential for scalable, memory-efficient language models and motivates further large-scale experimentation on our posit. The code for reproducing and pre-trained weights are available at \url{https://github.com/HMUNACHI/pete}.
\end{abstract}
\section{Introduction and Related Work}
Embedding layers, which are typically represented as 2D matrices of dimensions \(V \times d\) (where \(V\) is the vocabulary size and \(d\) the embedding dimension), often contribute more parameters in linguistic transformer models than other layers. Despite this, they do not necessarily yield proportional performance gains~\cite{lan2019albert, rajbhandari2020zero, shen2020qbert}. Several factors are thought to contribute to this inefficiency. 

Firstly, sparsity issues can lead to under-optimized embeddings since rare tokens remain underrepresented during training~\cite{svenstrup2017hash}. More importantly, embeddings may induce redundancy by assigning dense vectors to tokens with overlapping semantic roles, thereby wasting capacity~\cite{lan2019albert}. Additionally, traditional embeddings do not exploit entropy-driven compression; they allocate excessive parameters to frequent tokens without addressing the inherent redundancy in token distributions~\cite{shu2017compressing}.

Numerous research efforts have aimed to alleviate this parameter inefficiency by employing compression, adaptive designs, and alternative representations. Cai et al. \cite{cai2023dimensionlifting} highlighted the inefficiency of high-dimensional embedding representations in knowledge graph embeddings, proposing the Dimension Lifting Network (LiftNet). LiftNet replaces wide embeddings with a narrow embedding layer followed by a dimension lifting network, though this approach is not directly generalizable to all use cases. Xu et al. \cite{xu2023tensorgpt} introduced TensorGPT, which leverages Tensor-Train Decomposition to reduce the embedding layer's parameter count by up to 38.4 times with minimal performance degradation. In TensorGPT, a pre-trained embedding is projected to a lower dimension to facilitate deployment, even though a large embedding table must still be initially trained. Yan et al. \cite{yan2021adaptivemaskedtwins} proposed an Adaptively-Masked Twins-based Layer that dynamically adjusts embedding dimensions based on feature values, achieving significant memory savings and improved parameter utilization, albeit with a more complex training regime. Wang et al. \cite{wang2020structured} presented Structured Embedding Compression, which uses matrix factorization and product quantization to reduce the parameter count. Similarly, Lan et al. \cite{lan2019albert} developed ALBERT, a lightweight model that ties embeddings with the transformer's hidden representations to reduce redundancy. Shen et al. \cite{shen2020qbert} introduced Q-BERT, which employs Hessian-based quantization for embedding layers, though this method is compute-intensive. Additionally, hash-based embeddings have shown promise; for instance, Svenstrup et al. \cite{svenstrup2017hash} replaced traditional embedding tables with hash embeddings, achieving substantial parameter reductions for large vocabularies, albeit with specialized training procedures. Furthermore, although various forms of weight tying (such as embedding-output weight tying) are used in state-of-the-art models for language generation, such techniques do not readily benefit classification and language understanding tasks where output layers typically do not match the dimensions of the embedding layers. Also, sub-word tokenizations limits vocabulary size explosion, but we rather propose a solution where the vocabulary size is not an algorithmic complexity.

Despite these advances, further exploration is required to develop universally efficient and effective embedding mechanisms. In this work, we propose approximating the embedding layer with a combination of analytical methods and fewer learned parameters, opposed to learning embedding vectors for each token in the vocabulary. First motivation; Embedding layers in transformer models are fully factorized projections with not token inter-dependence, the semantic and pragmatic relationships are learned in the attention blocks. Second motivation; Byte-Pair Encoding (BPE) assigns token IDs based on each token's frequency in the corpus, effectively following a statistical pattern. The token ids carry distinguishing information. To this end, we argue that legacy embedding layers could be approximated by first transforming discrete token IDs into continuous values in the range \([-1, 1]\) (described in the next section), computing their Fourier expansions up to a predetermined degree \(n\) which corresponds to the desired embedding dimension, then projecting unto a more aligned vector space with a shared Multi-Layered Perceptron. Other polynomial bases (e.g., Chebyshev, Legendre, Taylor) could be employed, however Taylor polynomials require the computation of derivatives, Chebyshev polynomials exhibit auto-regressive properties, and Legendre polynomials involve factorial computations. These make them less amenable to optimizations on accelerators such as GPUs, TPUs and NPUs. Computing Fourier on the fly at inference is still computationally expensive compared to merely mapping token IDs to learned vectors, however they are very friendly hand-crafted hard-ware aware implementations, which we did in this work. 
\section{Methodology}
We adopt a parameter-efficient strategy to encode tokens by leveraging fixed Fourier basis functions combined with a learnable multilayer perceptron (MLP). This design is motivated by an information-theoretic perspective. In particular, BPE tokenization splits the input text into statistically significant word/subword units that are \emph{entropy-efficient} in terms of compression~\cite{sennrich2016neural}. Recall that for a token \(t\) with probability \(p(t)\), the \emph{surprisal} (self-information) is given by
\[
I(t) = -\log p(t),
\]
so that rare tokens (with low \(p(t)\)) contribute higher information. In many BPE schemes, token IDs are assigned in frequency order, meaning that frequent tokens receive lower IDs while rare tokens are assigned higher IDs. Although the token ID itself is an arbitrary label, its ordering reflects statistical properties of the vocabulary.
To harness this structure, we first normalize a token's integer ID \(p\) into the continuous interval \([-1,1]\) by defining
\[
x \;=\; 2\,\frac{p}{\texttt{vocabulary\_size}-1} \;-\; 1.
\]
This normalization maps discrete token IDs into a continuous, scale-invariant domain, allowing subsequent smooth transformations. Importantly, while the mapping is deterministic, it preserves the relative differences among tokens so that minor alterations (e.g., a word swap) affect the vector's magnitude more than its overall direction.
Next, we expand \(x\) into a high-dimensional embedding using a Fourier basis. For a chosen model dimension \(d_{\mathrm{model}}\), the token embedding \(\mathbf{T}(p) \in \mathbb{R}^{d_{\mathrm{model}}}\) is defined component-wise as
\[
T_{i}(p) 
\;=\;
\begin{cases}
\sin\Bigl(\bigl(\lfloor i/2 \rfloor + 1\bigr)\,\pi\,x\Bigr), & \text{if } i \text{ is even},\\[6pt]
\cos\Bigl(\bigl(\lfloor i/2 \rfloor + 1\bigr)\,\pi\,x\Bigr), & \text{if } i \text{ is odd}.
\end{cases}
\]
Here, lower-order Fourier terms (e.g., \(T_0(x)\) and \(T_1(x)\)) capture global, coarse-grained information, while higher-order terms (\(T_n(x)\) for \(n\geq2\)) encode finer details. Under the small-angle approximation, the difference between embeddings of adjacent tokens (i.e., \(p\) and \(p+1\)) is approximately
\[
\Delta x \;=\; \frac{2}{V-1},
\]
so that with a large vocabulary \(V\), adjacent token embeddings in raw Fourier space lie very close. Moreover, because BPE token assignment is not strictly semantically monotonic (e.g., the tokens ``cat'' and ``cathedral'' might receive consecutive IDs despite semantic differences), the Fourier expansion alone may not sufficiently differentiate tokens with similar IDs.
To mitigate this, we append a learnable MLP to the Fourier features. The final token representation is given by
\[
\mathbf{E}(p) \;=\; \text{MLP}\bigl(\mathbf{T}(p)\bigr) + \mathbf{T}(p).
\]
The role of the MLP is to \emph{stretch apart} tokens whose initial Fourier representations are too similar, ensuring that semantically distinct tokens (even if numerically adjacent) are mapped to adequately separated vectors for the attention mechanism. From the universal approximation viewpoint, given any continuous target embedding function \(f: [-1,1] \to \mathbb{R}\), the MLP can approximate the residual function \(H(z)=f\bigl(\varphi^{-1}(z)\bigr)-z\) (where \(\varphi(x)=\mathbf{T}(p)\)) uniformly. That is, for every \(\epsilon>0\) there exists an MLP \(M\) such that
\[
\sup_{x \in [-1, 1]} \Bigl|\, M\bigl(\varphi(x)\bigr) + \varphi(x) - f(x) \Bigr| < \epsilon.
\]
An information-theoretic perspective further illuminates this design via the Partial Information Decomposition (PID) framework~\cite{williams2010nonnegative, bertschinger2013quantifying}. In this view, the Fourier components of \(\mathbf{T}(p)\) serve as \emph{sources} of information about the token. Specifically:
\begin{itemize}
    \item \textbf{Unique Information} is carried directly by the Fourier expansion \(\mathbf{T}(p)\).
    \item \textbf{Redundant Information} represents overlapping aspects that appear both in \(\mathbf{T}(p)\) and in the MLP-transformed output.
    \item \textbf{Synergistic Information} emerges only when combining the Fourier features via the non-linear MLP.
\end{itemize}
The MLP is crucial for extracting higher-order interactions that a mere linear mapping could not achieve.
Finally, although dropout is commonly used to reduce overfitting, we observed that applying dropout to these normalized continuous mappings disrupts the smooth progression of token IDs and degrades performance as expected in Polynomial-based approximations. The continuous normalization itself appears to provide a regularizing effect, contributing to the minimal overfitting observed even in over-parameterized regimes.
\section{Experimentation and Results}
Due to resource constraints in both compute (a single Nvidia RTX 4090), team size and time limitations, our experiments are intentionally scaled down to a proof-of-concept design, and not optimized for main tracks at top conferences. Different neural architectures require different optimal hyper-parameters, but we evaluate the proposed transformer with Fourier embeddings (denoted as Fourier embeddings henceforth) and the corresponding baseline transformer under identical settings (established setups for transformers), differing only in the embedding layer. In the baseline transformer, the token embeddings are learned conventionally. Although extensive pre-training and evaluation at much larger model sized, on more benchmarks (e.g., GLUE or specialized reasoning tasks) would provide deeper insights, we defer these directions to future work. Our focus here is to assess how effectively an attention-based model can learn semantic information using a semi-approximated embedding mechanism.

Our experimental setup employs mixed-precision training on the entailment subsets of the SNLI and MNLI datasets~\cite{snli,mnli} with a batch size of 128 over 122,700 iterations, a learning rate of \(2\times10^{-5}\), and 1,000 warmup steps. For the baseline transformer, we use a dropout probability of 0.1 (yeiled best results), whereas Fourier embedding omits dropout. Fourier embedding implements its embedding mechanism via a custom CUDA kernel that fuses normalization and Fourier expansion. We utilize the BERT Tokenizer~\cite{devlin2019bert} (vocabulary size 30,522), rotary positional encoding~\cite{su2021rotary}, root-mean-squared layer normalization~\cite{liu2020rmsnorm}, and GeGLU activation~\cite{shazeer2020glu}. In addition, we employ average pooling and the AdamW optimizer~\cite{loshchilov2019decoupled}. Following the Fourier expansion, we include a position-wise feed-forward block with an intermediate expansion factor of 4 and GeGLU activation. While a large MLP might nearly match the parameter count of a learned embedding matrix, our experiments indicate that replacing this block with a simple linear layer only marginally degrades performance, while further reducing model size. Our primary objective is to isolate the impact of substituting the learned embedding layer with a deterministic alternative.

For training, we employ a contrastive loss function inspired by CLIP~\cite{radford2021learning} and InfoNCE~\cite{oord2018representation}. This loss encourages embeddings corresponding to matching sentence pairs (or entailment pairs) to be close in the embedding space, while non-matching pairs are pushed apart. A learnable temperature parameter is used to appropriately scale the cosine similarity scores. This contrastive framework leverages the entailment data (approximately 200k samples) to enforce semantic consistency in the learned representations.
Table~\ref{tab:model_scores} below summarizes our main findings.
\begin{table}[h!]
    \centering
    \begin{adjustbox}{max width=\textwidth}
    \begin{tabular}{|c|c|c|c|c|c|c|}
        \hline
        \textbf{Model} & \textbf{Layers/Heads} & \textbf{d-model} & \textbf{Params} & \textbf{STSB Spearman-R} & \textbf{STSB Pearson-R} & \textbf{Training Time}\\
        \hline
        Transformer (Fourier Embedding) & 1 & 256 & 1.1m & 74.93 & 74.54 & 37.88 min\\
        Transformer & 1 & 256 & 8.9m & 77.01 & 76.80 & 48.48 min\\
        \hline
        Transformer (Fourier Embedding) & 1 & 512 & 4.7m & 75.21 & 74.65 & 1.349 hr\\
        Transformer & 1 & 512 & 20.1m & 77.50 & 76.78 & 1.688 hr\\
        \hline
        Transformer (Fourier Embedding) & 2 & 256 & 2.2m & 76.38 & 76.02 & 1.009 hr\\
        Transformer & 2 & 256 & 9.9m & 77.34 & 76.89 & 1.322 hr\\
        \hline
        Transformer (Fourier Embedding) & 2 & 512 & 8.9m & 77.40 & 77.11 & 2.27 hr\\
        Transformer & 2 & 512 & 24.3m & 77.54 & 76.96 & 2.982 hr\\
        \hline
    \end{tabular}
    \end{adjustbox}
    \caption{STS-B validation scores without fine-tuning on STS-B.}
    \label{tab:model_scores}
\end{table}

With sufficient capacity (i.e., an adequately chosen \(d_{\text{model}}\) and number of layers), the Fourier-based embedding can yield representations as effective as those of learned embeddings. This finding supports the claim that semantic information can be efficiently captured by a deterministic, parameter-free mapping when combined with appropriate downstream processing. In particular, our results show that at two layers/heads with dimensions of 256 or 512, Fourier embedding converges to performance levels comparable to a standard transformer. By contrast, transformers of the same size appear over-parameterized, offering no measurable performance gains while introducing unnecessary parameters. Conversely, models quickly become under-parameterized when they use fewer than two layers/heads or a \(d_{\text{model}}\) below 256. Another advantage of Fourier embedding is its avoidance of the computational overhead associated with large embedding tables. Our custom CUDA kernel, which fuses normalization and Fourier expansion, likely contributes to reduced training times. 

While these improvements are not strictly proportional to model size, they may become substantial in large language-modeling scenarios. Moreover, Fourier embedding allows for flexible downscaling without sacrificing the capacity to learn meaningful relationships, since it does not rely on the vocabulary size. When parameter counts are held constant, Fourier embedding offers notably better performance than transformers by freeing up capacity for deeper attention layers. Traditional transformers allocate a large fraction of their parameters to the embedding layer, which grows with the vocabulary size. In contrast, our method replaces this with a deterministic Fourier expansion, combined with a relatively small MLP, eliminating the need for a heavy, learned embedding table. This shift effectively "frees up" parameters that can then be invested elsewhere in the network, leading to more balanced and efficient architectures.

Ultimately, parameter size percentage reductions slows down as transformer layers are scaled horizontally (adding more layers), hence we experimented with further parameter reduction by transforming the intermediate dimensions of the MLP blocks from dim x 4 to dim / 4, equivalent to using lower rank weight matrices. Table~\ref{tab:model_comparisons} below summarizes the performance of these much smaller Parameter-Efficient Transformer Embeddings (PETE) when trained with the same regime and fine-tuned on STS-B.

\begin{table}[h!]
    \centering
    \begin{adjustbox}{max width=\textwidth}
    \begin{tabular}{|c|c|c|c|c|c|c|}
        \hline
        \textbf{Model} & \textbf{Params} & \textbf{STSB Pearson-R} & \textbf{STSB Spearman-R}\\
        \hline
        PETE & 58k+ & 69.0 & 69.5\\
        \hline
        PETE & 396k+ & 78.0 & 78.0\\
        \hline
        PETE & 1.5m+ & 79.7 & 79.7\\
        \hline
        PETE & 3.6m+ & \textbf{81.7} & \textbf{81.9}\\
        \hline
        BERT-Tiny (official report) & 4m+ & 74.3 & 73.6\\
        \hline
        BERT-Mini (official report) & 11m+ & 74.3 & 73.6\\
        \hline
        TinyBERT (official report) & 14.5m & -- & 80.4\\
        \hline 
        MobileBert-Tiny (official report) & 15.1m  &  -- & 80.1\\
        \hline
        BERT-Small (official report) & 29m+ & 78.8 & 77.0\\
        \hline
    \end{tabular}
    \end{adjustbox}
    \caption{STS-B validation scores after fine-tuning on STS-B against tiny models}
    \label{tab:model_comparisons}
\end{table}
\section{Discussion and Conclusion}
The experimental results support our central claim: a Fourier-based, parameter-free embedding scheme can yield competitive performance compared to traditional learned embeddings while substantially reducing both the parameter count and training time. In configurations with equal overall parameters, the PETE even outperforms the traditional transformer, owing to its efficient allocation of resources. By eliminating large, learned embedding tables, our method frees up capacity for deeper attention layers and better overall parameter utilization.

It is important to note that these experiments were conducted on relatively small-scale models and focused on natural language inference and the STS-B task. Whether the benefits extend to larger-scale models and more diverse tasks such as large-scale reasoning remains an open question for future work. In particular, scaling up to vocabularies in the hundreds of thousands may pose challenges, as a dense coverage of the continuous interval could lead to near-colliding embeddings. Future research should investigate whether attention blocks can effectively separate these near-colliding embeddings and whether the deterministic Fourier expansion can capture discrete lexical phenomena such as morphological variants, homonyms, or polysemy. Another promising direction is the exploration of alternative polynomial bases (e.g., Chebyshev or Legendre) for embedding mechanisms. Although these alternatives may offer theoretical advantages, our findings suggest that the Fourier basis is particularly well-suited for optimization on modern accelerators such as GPUs and TPUs. Its analytic form not only contributes to memory and computational efficiency but also opens avenues for improved interpretability of token embeddings. For example, by examining how final representations evolve as a function of token ID, one may gain insights into the model's internal semantic organization.
In conclusion, our work demonstrates that a deterministic, Fourier-based token embedding, when paired with appropriate downstream processing, can serve as an efficient alternative to learned embedding tables. This approach not only reduces the parameter burden but also reallocates capacity more effectively within the network, potentially leading to more robust and scalable models.
\documentclass{article}
\usepackage{style,times}
\usepackage{booktabs}
\usepackage{caption}
\usepackage{array}
\usepackage{graphicx}
\usepackage{array}
\usepackage{booktabs}
\usepackage{adjustbox}
\usepackage{amsmath, amsthm, amssymb}
\newtheorem{theorem}{Theorem}
\newtheorem{lemma}{Lemma}
\usepackage{hyperref}
\usepackage{url}
\begin{document}
\begin{center}
\title{Parameter-Efficient Transformer Embedding}
\end{center}
\author{Henry Ndubuaku \\ 
Cactus Compute, Inc. \\
London, United Kingdom \\
\texttt{henry@cactuscompute.com} \\
\And
Mouad Talhi \\
Department of Computing, Imperial College \\
London, United Kingdom \\
\texttt{mt924@ic.ac.uk} \\
}
\maketitle
\begin{abstract}
Embedding layers in transformer-based language models typically account for the largest share of model parameters, scaling with vocabulary size but not yielding performance gains linear to scale. We propose an alternative approach in which token embedding vectors are first generated deterministically, directly from the token IDs using a Fourier expansion of their normalized values, followed by a lightweight multilayer perceptron (MLP) that captures higher-order interactions. We train standard transformers with and our architecture on natural language inference tasks (SNLI and MNLI), and evaluate zero-shot performance on sentence textual similarity (STS-B). Our results demonstrate that the proposed method achieves competitive performance using significantly fewer parameters, trains faster, and generalizes well without the need for dropout. This proof-of-concept study highlights the potential for scalable, memory-efficient language models and motivates further large-scale experimentation on our posit. The code for reproducing and pre-trained weights are available at \url{https://github.com/HMUNACHI/pete}.
\end{abstract}
\section{Introduction and Related Work}
Embedding layers, which are typically represented as 2D matrices of dimensions \(V \times d\) (where \(V\) is the vocabulary size and \(d\) the embedding dimension), often contribute more parameters in linguistic transformer models than other layers. Despite this, they do not necessarily yield proportional performance gains~\cite{lan2019albert, rajbhandari2020zero, shen2020qbert}. Several factors are thought to contribute to this inefficiency. 

Firstly, sparsity issues can lead to under-optimized embeddings since rare tokens remain underrepresented during training~\cite{svenstrup2017hash}. More importantly, embeddings may induce redundancy by assigning dense vectors to tokens with overlapping semantic roles, thereby wasting capacity~\cite{lan2019albert}. Additionally, traditional embeddings do not exploit entropy-driven compression; they allocate excessive parameters to frequent tokens without addressing the inherent redundancy in token distributions~\cite{shu2017compressing}.

Numerous research efforts have aimed to alleviate this parameter inefficiency by employing compression, adaptive designs, and alternative representations. Cai et al. \cite{cai2023dimensionlifting} highlighted the inefficiency of high-dimensional embedding representations in knowledge graph embeddings, proposing the Dimension Lifting Network (LiftNet). LiftNet replaces wide embeddings with a narrow embedding layer followed by a dimension lifting network, though this approach is not directly generalizable to all use cases. Xu et al. \cite{xu2023tensorgpt} introduced TensorGPT, which leverages Tensor-Train Decomposition to reduce the embedding layer's parameter count by up to 38.4 times with minimal performance degradation. In TensorGPT, a pre-trained embedding is projected to a lower dimension to facilitate deployment, even though a large embedding table must still be initially trained. Yan et al. \cite{yan2021adaptivemaskedtwins} proposed an Adaptively-Masked Twins-based Layer that dynamically adjusts embedding dimensions based on feature values, achieving significant memory savings and improved parameter utilization, albeit with a more complex training regime. Wang et al. \cite{wang2020structured} presented Structured Embedding Compression, which uses matrix factorization and product quantization to reduce the parameter count. Similarly, Lan et al. \cite{lan2019albert} developed ALBERT, a lightweight model that ties embeddings with the transformer's hidden representations to reduce redundancy. Shen et al. \cite{shen2020qbert} introduced Q-BERT, which employs Hessian-based quantization for embedding layers, though this method is compute-intensive. Additionally, hash-based embeddings have shown promise; for instance, Svenstrup et al. \cite{svenstrup2017hash} replaced traditional embedding tables with hash embeddings, achieving substantial parameter reductions for large vocabularies, albeit with specialized training procedures. Furthermore, although various forms of weight tying (such as embedding-output weight tying) are used in state-of-the-art models for language generation, such techniques do not readily benefit classification and language understanding tasks where output layers typically do not match the dimensions of the embedding layers. Also, sub-word tokenizations limits vocabulary size explosion, but we rather propose a solution where the vocabulary size is not an algorithmic complexity.

Despite these advances, further exploration is required to develop universally efficient and effective embedding mechanisms. In this work, we propose approximating the embedding layer with a combination of analytical methods and fewer learned parameters, opposed to learning embedding vectors for each token in the vocabulary. First motivation; Embedding layers in transformer models are fully factorized projections with not token inter-dependence, the semantic and pragmatic relationships are learned in the attention blocks. Second motivation; Byte-Pair Encoding (BPE) assigns token IDs based on each token's frequency in the corpus, effectively following a statistical pattern. The token ids carry distinguishing information. To this end, we argue that legacy embedding layers could be approximated by first transforming discrete token IDs into continuous values in the range \([-1, 1]\) (described in the next section), computing their Fourier expansions up to a predetermined degree \(n\) which corresponds to the desired embedding dimension, then projecting unto a more aligned vector space with a shared Multi-Layered Perceptron. Other polynomial bases (e.g., Chebyshev, Legendre, Taylor) could be employed, however Taylor polynomials require the computation of derivatives, Chebyshev polynomials exhibit auto-regressive properties, and Legendre polynomials involve factorial computations. These make them less amenable to optimizations on accelerators such as GPUs, TPUs and NPUs. Computing Fourier on the fly at inference is still computationally expensive compared to merely mapping token IDs to learned vectors, however they are very friendly hand-crafted hard-ware aware implementations, which we did in this work. 
\section{Methodology}
We adopt a parameter-efficient strategy to encode tokens by leveraging fixed Fourier basis functions combined with a learnable multilayer perceptron (MLP). This design is motivated by an information-theoretic perspective. In particular, BPE tokenization splits the input text into statistically significant word/subword units that are \emph{entropy-efficient} in terms of compression~\cite{sennrich2016neural}. Recall that for a token \(t\) with probability \(p(t)\), the \emph{surprisal} (self-information) is given by
\[
I(t) = -\log p(t),
\]
so that rare tokens (with low \(p(t)\)) contribute higher information. In many BPE schemes, token IDs are assigned in frequency order, meaning that frequent tokens receive lower IDs while rare tokens are assigned higher IDs. Although the token ID itself is an arbitrary label, its ordering reflects statistical properties of the vocabulary.
To harness this structure, we first normalize a token's integer ID \(p\) into the continuous interval \([-1,1]\) by defining
\[
x \;=\; 2\,\frac{p}{\texttt{vocabulary\_size}-1} \;-\; 1.
\]
This normalization maps discrete token IDs into a continuous, scale-invariant domain, allowing subsequent smooth transformations. Importantly, while the mapping is deterministic, it preserves the relative differences among tokens so that minor alterations (e.g., a word swap) affect the vector's magnitude more than its overall direction.
Next, we expand \(x\) into a high-dimensional embedding using a Fourier basis. For a chosen model dimension \(d_{\mathrm{model}}\), the token embedding \(\mathbf{T}(p) \in \mathbb{R}^{d_{\mathrm{model}}}\) is defined component-wise as
\[
T_{i}(p) 
\;=\;
\begin{cases}
\sin\Bigl(\bigl(\lfloor i/2 \rfloor + 1\bigr)\,\pi\,x\Bigr), & \text{if } i \text{ is even},\\[6pt]
\cos\Bigl(\bigl(\lfloor i/2 \rfloor + 1\bigr)\,\pi\,x\Bigr), & \text{if } i \text{ is odd}.
\end{cases}
\]
Here, lower-order Fourier terms (e.g., \(T_0(x)\) and \(T_1(x)\)) capture global, coarse-grained information, while higher-order terms (\(T_n(x)\) for \(n\geq2\)) encode finer details. Under the small-angle approximation, the difference between embeddings of adjacent tokens (i.e., \(p\) and \(p+1\)) is approximately
\[
\Delta x \;=\; \frac{2}{V-1},
\]
so that with a large vocabulary \(V\), adjacent token embeddings in raw Fourier space lie very close. Moreover, because BPE token assignment is not strictly semantically monotonic (e.g., the tokens ``cat'' and ``cathedral'' might receive consecutive IDs despite semantic differences), the Fourier expansion alone may not sufficiently differentiate tokens with similar IDs.
To mitigate this, we append a learnable MLP to the Fourier features. The final token representation is given by
\[
\mathbf{E}(p) \;=\; \text{MLP}\bigl(\mathbf{T}(p)\bigr) + \mathbf{T}(p).
\]
The role of the MLP is to \emph{stretch apart} tokens whose initial Fourier representations are too similar, ensuring that semantically distinct tokens (even if numerically adjacent) are mapped to adequately separated vectors for the attention mechanism. From the universal approximation viewpoint, given any continuous target embedding function \(f: [-1,1] \to \mathbb{R}\), the MLP can approximate the residual function \(H(z)=f\bigl(\varphi^{-1}(z)\bigr)-z\) (where \(\varphi(x)=\mathbf{T}(p)\)) uniformly. That is, for every \(\epsilon>0\) there exists an MLP \(M\) such that
\[
\sup_{x \in [-1, 1]} \Bigl|\, M\bigl(\varphi(x)\bigr) + \varphi(x) - f(x) \Bigr| < \epsilon.
\]
An information-theoretic perspective further illuminates this design via the Partial Information Decomposition (PID) framework~\cite{williams2010nonnegative, bertschinger2013quantifying}. In this view, the Fourier components of \(\mathbf{T}(p)\) serve as \emph{sources} of information about the token. Specifically:
\begin{itemize}
    \item \textbf{Unique Information} is carried directly by the Fourier expansion \(\mathbf{T}(p)\).
    \item \textbf{Redundant Information} represents overlapping aspects that appear both in \(\mathbf{T}(p)\) and in the MLP-transformed output.
    \item \textbf{Synergistic Information} emerges only when combining the Fourier features via the non-linear MLP.
\end{itemize}
The MLP is crucial for extracting higher-order interactions that a mere linear mapping could not achieve.
Finally, although dropout is commonly used to reduce overfitting, we observed that applying dropout to these normalized continuous mappings disrupts the smooth progression of token IDs and degrades performance as expected in Polynomial-based approximations. The continuous normalization itself appears to provide a regularizing effect, contributing to the minimal overfitting observed even in over-parameterized regimes.
\section{Experimentation and Results}
Due to resource constraints in both compute (a single Nvidia RTX 4090), team size and time limitations, our experiments are intentionally scaled down to a proof-of-concept design, and not optimized for main tracks at top conferences. Different neural architectures require different optimal hyper-parameters, but we evaluate the proposed transformer with Fourier embeddings (denoted as Fourier embeddings henceforth) and the corresponding baseline transformer under identical settings (established setups for transformers), differing only in the embedding layer. In the baseline transformer, the token embeddings are learned conventionally. Although extensive pre-training and evaluation at much larger model sized, on more benchmarks (e.g., GLUE or specialized reasoning tasks) would provide deeper insights, we defer these directions to future work. Our focus here is to assess how effectively an attention-based model can learn semantic information using a semi-approximated embedding mechanism.

Our experimental setup employs mixed-precision training on the entailment subsets of the SNLI and MNLI datasets~\cite{snli,mnli} with a batch size of 128 over 122,700 iterations, a learning rate of \(2\times10^{-5}\), and 1,000 warmup steps. For the baseline transformer, we use a dropout probability of 0.1 (yeiled best results), whereas Fourier embedding omits dropout. Fourier embedding implements its embedding mechanism via a custom CUDA kernel that fuses normalization and Fourier expansion. We utilize the BERT Tokenizer~\cite{devlin2019bert} (vocabulary size 30,522), rotary positional encoding~\cite{su2021rotary}, root-mean-squared layer normalization~\cite{liu2020rmsnorm}, and GeGLU activation~\cite{shazeer2020glu}. In addition, we employ average pooling and the AdamW optimizer~\cite{loshchilov2019decoupled}. Following the Fourier expansion, we include a position-wise feed-forward block with an intermediate expansion factor of 4 and GeGLU activation. While a large MLP might nearly match the parameter count of a learned embedding matrix, our experiments indicate that replacing this block with a simple linear layer only marginally degrades performance, while further reducing model size. Our primary objective is to isolate the impact of substituting the learned embedding layer with a deterministic alternative.

For training, we employ a contrastive loss function inspired by CLIP~\cite{radford2021learning} and InfoNCE~\cite{oord2018representation}. This loss encourages embeddings corresponding to matching sentence pairs (or entailment pairs) to be close in the embedding space, while non-matching pairs are pushed apart. A learnable temperature parameter is used to appropriately scale the cosine similarity scores. This contrastive framework leverages the entailment data (approximately 200k samples) to enforce semantic consistency in the learned representations.
Table~\ref{tab:model_scores} below summarizes our main findings.
\begin{table}[h!]
    \centering
    \begin{adjustbox}{max width=\textwidth}
    \begin{tabular}{|c|c|c|c|c|c|c|}
        \hline
        \textbf{Model} & \textbf{Layers/Heads} & \textbf{d-model} & \textbf{Params} & \textbf{STSB Spearman-R} & \textbf{STSB Pearson-R} & \textbf{Training Time}\\
        \hline
        Transformer (Fourier Embedding) & 1 & 256 & 1.1m & 74.93 & 74.54 & 37.88 min\\
        Transformer & 1 & 256 & 8.9m & 77.01 & 76.80 & 48.48 min\\
        \hline
        Transformer (Fourier Embedding) & 1 & 512 & 4.7m & 75.21 & 74.65 & 1.349 hr\\
        Transformer & 1 & 512 & 20.1m & 77.50 & 76.78 & 1.688 hr\\
        \hline
        Transformer (Fourier Embedding) & 2 & 256 & 2.2m & 76.38 & 76.02 & 1.009 hr\\
        Transformer & 2 & 256 & 9.9m & 77.34 & 76.89 & 1.322 hr\\
        \hline
        Transformer (Fourier Embedding) & 2 & 512 & 8.9m & 77.40 & 77.11 & 2.27 hr\\
        Transformer & 2 & 512 & 24.3m & 77.54 & 76.96 & 2.982 hr\\
        \hline
    \end{tabular}
    \end{adjustbox}
    \caption{STS-B validation scores without fine-tuning on STS-B.}
    \label{tab:model_scores}
\end{table}

With sufficient capacity (i.e., an adequately chosen \(d_{\text{model}}\) and number of layers), the Fourier-based embedding can yield representations as effective as those of learned embeddings. This finding supports the claim that semantic information can be efficiently captured by a deterministic, parameter-free mapping when combined with appropriate downstream processing. In particular, our results show that at two layers/heads with dimensions of 256 or 512, Fourier embedding converges to performance levels comparable to a standard transformer. By contrast, transformers of the same size appear over-parameterized, offering no measurable performance gains while introducing unnecessary parameters. Conversely, models quickly become under-parameterized when they use fewer than two layers/heads or a \(d_{\text{model}}\) below 256. Another advantage of Fourier embedding is its avoidance of the computational overhead associated with large embedding tables. Our custom CUDA kernel, which fuses normalization and Fourier expansion, likely contributes to reduced training times. 

While these improvements are not strictly proportional to model size, they may become substantial in large language-modeling scenarios. Moreover, Fourier embedding allows for flexible downscaling without sacrificing the capacity to learn meaningful relationships, since it does not rely on the vocabulary size. When parameter counts are held constant, Fourier embedding offers notably better performance than transformers by freeing up capacity for deeper attention layers. Traditional transformers allocate a large fraction of their parameters to the embedding layer, which grows with the vocabulary size. In contrast, our method replaces this with a deterministic Fourier expansion, combined with a relatively small MLP, eliminating the need for a heavy, learned embedding table. This shift effectively "frees up" parameters that can then be invested elsewhere in the network, leading to more balanced and efficient architectures.

Ultimately, parameter size percentage reductions slows down as transformer layers are scaled horizontally (adding more layers), hence we experimented with further parameter reduction by transforming the intermediate dimensions of the MLP blocks from dim x 4 to dim / 4, equivalent to using lower rank weight matrices. Table~\ref{tab:model_comparisons} below summarizes the performance of these much smaller Parameter-Efficient Transformer Embeddings (PETE) when trained with the same regime and fine-tuned on STS-B.

\begin{table}[h!]
    \centering
    \begin{adjustbox}{max width=\textwidth}
    \begin{tabular}{|c|c|c|c|c|c|c|}
        \hline
        \textbf{Model} & \textbf{Params} & \textbf{STSB Pearson-R} & \textbf{STSB Spearman-R}\\
        \hline
        PETE & 58k+ & 69.0 & 69.5\\
        \hline
        PETE & 396k+ & 78.0 & 78.0\\
        \hline
        PETE & 1.5m+ & 79.7 & 79.7\\
        \hline
        PETE & 3.6m+ & \textbf{81.7} & \textbf{81.9}\\
        \hline
        BERT-Tiny (official report) & 4m+ & 74.3 & 73.6\\
        \hline
        BERT-Mini (official report) & 11m+ & 74.3 & 73.6\\
        \hline
        TinyBERT (official report) & 14.5m & -- & 80.4\\
        \hline 
        MobileBert-Tiny (official report) & 15.1m  &  -- & 80.1\\
        \hline
        BERT-Small (official report) & 29m+ & 78.8 & 77.0\\
        \hline
    \end{tabular}
    \end{adjustbox}
    \caption{STS-B validation scores after fine-tuning on STS-B against tiny models}
    \label{tab:model_comparisons}
\end{table}
\section{Discussion and Conclusion}
The experimental results support our central claim: a Fourier-based, parameter-free embedding scheme can yield competitive performance compared to traditional learned embeddings while substantially reducing both the parameter count and training time. In configurations with equal overall parameters, the PETE even outperforms the traditional transformer, owing to its efficient allocation of resources. By eliminating large, learned embedding tables, our method frees up capacity for deeper attention layers and better overall parameter utilization.

It is important to note that these experiments were conducted on relatively small-scale models and focused on natural language inference and the STS-B task. Whether the benefits extend to larger-scale models and more diverse tasks such as large-scale reasoning remains an open question for future work. In particular, scaling up to vocabularies in the hundreds of thousands may pose challenges, as a dense coverage of the continuous interval could lead to near-colliding embeddings. Future research should investigate whether attention blocks can effectively separate these near-colliding embeddings and whether the deterministic Fourier expansion can capture discrete lexical phenomena such as morphological variants, homonyms, or polysemy. Another promising direction is the exploration of alternative polynomial bases (e.g., Chebyshev or Legendre) for embedding mechanisms. Although these alternatives may offer theoretical advantages, our findings suggest that the Fourier basis is particularly well-suited for optimization on modern accelerators such as GPUs and TPUs. Its analytic form not only contributes to memory and computational efficiency but also opens avenues for improved interpretability of token embeddings. For example, by examining how final representations evolve as a function of token ID, one may gain insights into the model's internal semantic organization.
In conclusion, our work demonstrates that a deterministic, Fourier-based token embedding, when paired with appropriate downstream processing, can serve as an efficient alternative to learned embedding tables. This approach not only reduces the parameter burden but also reallocates capacity more effectively within the network, potentially leading to more robust and scalable models.
\input{main.bbl}
\end{document}
\typeout{get arXiv to do 4 passes: Label(s) may have changed. Rerun}

\end{document}
\typeout{get arXiv to do 4 passes: Label(s) may have changed. Rerun}

\end{document}
\typeout{get arXiv to do 4 passes: Label(s) may have changed. Rerun}

\end{document}
\typeout{get arXiv to do 4 passes: Label(s) may have changed. Rerun}
